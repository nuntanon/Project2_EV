\chapter{บทสรุปและข้อเสนอแนะ}
\section{สรุปผลการทดสอบแบตเตอรี่}
จากการทดสอบแบตเตอรี่ในหัวข้อการทดสอบการป้องกันการชาร์จเกินจากมาตรฐาน UN ECE R136 จะเห็นได้ว่าสำหรับแบตเตอรี่สำหรับรถจักยานยนต์ไฟฟ้า 72V30Ah โมดูลนี้ผ่านการทดสอบไปได้
โดยที่ไม่เกิดความเสียหายกับแบตเตอรี่โมดูลนี้เนื่องจากแบตเตอรี่โมดูลนี้มีระบบการจัดการแบตเตอรี่(BMS)ที่ทำหน้าที่ป้องกันแบตเตอรี่ไม่ให้มีการชาร์จเกินแรงดันสูงสุดของแบตเตอรี่และขัดจังหวะการชาร์จระหว่างการทดสอบ
จึงทำให้ผ่านการทดสอบด้วยดีสำหรับแบตเตอรี่สำหรับรถสามล้อไฟฟ้า 72V60Ah จากการทดสอบการป้องกันการชาร์จเกินนี้เนื่องจากไม่ได้ทำการทดสอบที่ถูกต้องซึ่งเกิดจากการที่ทำการชาร์จโดยไม่ได้ชาร์จ
ตามที่ผู้ผลิตกำหนดโดยใช้ช่องทางการชาร์จเดียวกับช่องทางการดิสชาร์จซึ่งทางผู้ผลิตได้กำหนดให้ทำการชาร์จตรงช่องทางการชาร์จเช่นเดียวกันกับการดิสชาร์จผู้ผลิตกำหนดให้ดิสชาร์จตรงช่องทางการดิสชาร์จโดยผลการทดลองพบว่าข้อมูล
ไม่เพียงพอต่อการสรุปว่าผ่านการทดสอบหรือไม่เนื่องจากเหตุผลข้างต้นสำหรับแบตเตอรี่ 72V72Ah หลังจากการทดสอบการป้องกันการชาร์จเกินพบว่าผ่านการทดสอบเช่นเดียวกับแบตเตอรี่สำหรับรถจักรยานยนต์ไฟฟ้า
โดยที่มีระบบการจัดการแบตเตอรี่(BMS)ขัดจังหวะการทดสอบเพื่อป้องกันแรงดันของแบตเตอรี่เกินกว่าที่กำหนด
\newline
\hspace*{2cm}
และจากการทดสอบแบตเตอรี่ในหัวข้อการทดสอบการป้องกันการดิสชาร์จเกินจากมาตรฐาน UN ECE R136 จะเห็นได้ว่าสำหรับแบตเตอรี่สำหรับรถจักยานยนต์ไฟฟ้า 72V30Ah โมดูลนี้ผ่านการทดสอบไปได้
โดยที่ไม่เกิดความเสียหายกับแบตเตอรี่โมดูลนี้เนื่องจากแบตเตอรี่โมดูลนี้มีระบบการจัดการแบตเตอรี่(BMS)ป้องกันไม่ให้แบตเตอรี่แรงดันต่ำเกินกว่าที่กำหนดซึ่งการทดสอบถูกขัดจังหวะโดยระบบการจัดการแบตเตอรี่สำหรับ
แบตเตอรี่สำหรับรถสามล้อไฟฟ้า 72V60Ah เช่นเดียวกันกับแบตเตอรี่สำหรับรถจักรยานยนต์ไฟฟ้าแบตเตอรี่โมดูลนี้ผ่านการทดสอบโดยมีระบบการจัดการแบตเตอรี่ขัดจังหวะการทดสอบและสุดท้ายสำหรับแบตเตอรี่ 72V72Ah
เช่นเดียวกันกับการทดสอบแบตเตอรี่โมดูลที่ผ่านแบตเตอรี่โมดูลนี้ผ่านการทดสอบโดยมีระบบการจัดการแบตเตอรี่ขัดจังหวะการทดสอบทำให้ผ่านการทดสอบโดยไม่เกิดความเสียหายกับแบตเตอรี่เช่นกัน และจะเห็นได้ว่า
แบตเตอรี่ที่ผ่านการทดสอบทั้งการป้องกันการชาร์จเกินและการป้องกันการดิสชาร์จเกินผ่านการทดสอบเนื่องจากมีระบบป้องกันไม่ให้โมดูลแบตเตอรี่มีแรงดันต่ำเกินและสูงเกินในกรณีที่ไม่มีระบบป้องกันนี้จึงไม่สามารถทราบได้ว่าผลการทดสอบ
จะเป็นอย่างไร
\newline
\hspace*{2cm}
จากการทดสอบการวัดความต้านทานภายในในการทดสอบสามารถทดสอบจะทดสอบเฉพาะโมดูลแบตเตอรี่ 72V72Ah เนื่องจากทราบคุณสมบัติชัดเจนทำให้มีความปลอดภัยต่อการทดสอบซึ่งจากการทดสอบพบว่าขณะที่แบตเตอรี่
แรงดันขณะดิสชาร์จที่อัตรากระแส 0.2C โมดูลแบตเตอรี่นี้มีแรงดัน 72V และขณะที่ดิสชาร์จที่อัตรากระแส 1C แรงดันลดลงอย่างเรวดเร็วซึ่งอยู่ที่ 70.4V จากข้อมูลนี้จึงทำให้ได้ค่าความต้านทานภายในอยู่ที่
$26.8 m\Omega $ ทั้งนี้ความต้านทานภายในเปลี่ยนแปลงไปตามแรงดัน ณ ขณะที่ทดสอบซึ่งการทดสอบนี้ทดสอบที่แรงดันค่าหนึ่งเท่านั้นนั่นก็คือ 72V
\newline
\hspace*{2cm}
จากการทดสอบเพื่อหาระยะเวลาพักแบตเตอรี่ที่แบตเตอรี่นั้นจะไม่เกิดการเปลี่ยนแปลงซึ่งจะเห็นได้ว่าแรงดันเริ่มที่จะหยุดเปลี่ยนแปลงเมื่อทำการพักมาแล้วเป็นระยะเวลา 20 นาทีซึ่งสรุปได้ว่าระยะเวลานี้เป็นระยะเวลาพักที่เหมาะสมก่อนที่จะทำการ
ทดสอบอย่างอื่นต่อไปเพื่อความแม่นยำในการทดสอบต่างๆและสุดท้ายจากการทดสอบอัตรากระแสมีผลต่ออัตรากระแสนี้จะเห็นได้ว่าเมื่ออัตรากระแสเปลี่ยนแปลงไปทำให้แรงดันต่อช่วงความจุนั้นเปลี่ยนแปลงซึ่งถ้าหากใช้อัตรากระแสที่มาก
จะทำให้แรงดันมีการเปลี่ยนแปลงที่รวดเร็วกว่าการใช้อัตรากระแสที่น้อยกว่า
%================================================================================
\section{แนวทางการแก้ปัญหา}
เนื่องจากโมดูลแบตเตอรี่สำหรับรถสามล้อไฟฟ้านั้นไม่ทราบคุณสมบัติโดยละเอียดซึ่งทราบเพียงพิกัดและชนิดของแบตเตอรี่เพียงเท่านั้นและจากการที่ช่องทางการชาร์จของโมดูลแบตเตอรี่นี้มีระบบการจัดการหรือระบบป้องกันที่
ทางผู้ผลิตได้ออกแบบไว้ซึ่งสามารถใช้ได้เฉพาะเครื่องชาร์จที่ผู้ผลิตได้กำหนดเอาไว้ซึ่งทางคณะผู้จัดทำไม่ทราบถึงการทำงานของระบบป้องกันหรือระบบการจัดการดังกล่าวทำให้ไม่สามารถทดสอบการชาร์จได้อย่างถูกต้องได้
เช่นเดียวกับโมดูลแบตเตอรี่สำหรับจักรยานยนต์ไฟฟ้า 72V30Ah ซึ่งไม่ทราบคุณลักษณะโดยละเอียดทำให้มีข้อจำกัดในการทดสอบโดยแนวทางการแก้ปัญหาทั้งสองนี้มีแนวทางการแก้ปัญหาคือ
\begin{itemize}
{\item ติดต่อสอบถามบริษัทผู้ผลิตหรือจำหน่ายสินค้าเพื่อขอรายละเอียดและคุณลักษณะโมดูลแบตเตอรี่เพิ่มเติม}
{\item ศึกษาเครื่องชาร์จโมดูลแบตเตอรี่สำหรับรถสามล้อไฟฟ้าและระบบป้องกันหรือระบบการจัดการของแบตเตอรี่โมดูลนี้เพิ่มเติม}
\end{itemize}
%================================================================================
\section{ข้อเสนอแนะ}
เนื่องจากอุปกรณ์ในการทดสอบนั้นไม่มีความพร้อมต่อการทดสอบมากเท่าที่ควรดังนั้นก่อนเริ่มการทดสอบควรศึกษาขั้นตอนการทดสอบอย่างละเอียดและตรวจสอบความพร้อมของอุปกรณ์และเครื่องมืออื่นๆที่ใช้สำหรับการทดสอบ
ก่อนเริ่มทำการทดสอบและควรกำหนดแผนการทดสอบก่อนการทดสอบอย่างชัดเจนทั้งนี้เพื่อความประหยัดเวลาและอาจจะลดขั้นตอนการทำงานบางอย่างที่ไม่จำเป็นได้












