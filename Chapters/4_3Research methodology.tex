\chapter{การทดสอบแบตเตอรี่ตามมาตรฐาน}
แบตเตอรี่เป็นส่วนประกอบที่มีความสำคัญมากสำหรับยานยนต์ไฟฟ้าเนื่องจากเป็นอุปกรณ์ที่กักเก็บและให้พลังงานไฟฟ้ากับยานยนต์ไฟฟ้าเพื่อความปลอดภัยของผู้ที่ใช้งานยานยนต์ไฟฟ้ามารตฐานต่างๆจึงถูกกำหนดขึ้นเพื่อ
ใช้กับทุกส่วนประกอบของยานยนต์ไฟฟ้ารวมถึงแบตเตอรี่ด้วยเช่นกันซึ่งการทดสอบแบตเตอรี่ที่ทางคณะผู้จัดทำได้ทำการทดสอบนั้นจะทดสอบตามมาตรฐาน UN ECE Regulation 136 ทั้งหมด 3 หัวข้อดังนี้
\subsection{การทดสอบการป้องกันการลัดวงจรภายนอกของแบตเตอรี่}
ในหัวข้อนี้จะเป็นการทดสอบการป้องกันการลัดวงจรภายนอกแบตเตอรี่ของแบตเตอรี่โดยจุดประสงค์ของการทดสอบนี้เพื่อทดสอบความสามารถการป้องกันการลัดวงจรของแบตเตอรี่โดยถ้าแบตเตอรี่มีอุปกรณ์ป้องกันการลัดวงจรอยู่ภาย
ในดังนั้นอุปกรณ์ป้องกันการลัดวงจรนี้ต้องขัดจังหวะหรือจำกัดกระแสลัดวงจรเพื่อป้องกันความเสียหายที่จะเกิดขึ้นจากการลัดวงจรของแบตเตอรี่
\newline
\newline
\textbf{เงื่อนไขทั่วไปในขั้นตอนการทดสอบ}
\begin{itemize}
{\item ระหว่างการทดสอบแบตเตอรี่ต้องทำงานอยู่ในอุณหภูมิ 20$\pm$10$^{\circ}C$ หรือสูงกว่า}
{\item ก่อนการทดสอบแบตเตอรี่ต้องมีระดับ SOC มากกว่า 50\% ของช่วง SOC ที่แบตเตอรี่อยู่ในสภาวะการทำงานปกติ}
{\item เมื่อเริ่มทำการทดสอบอุปกรณ์ป้องกันทุกอย่างที่ส่งผลต่อการทำงานของแบตเตอรี่ซึ่งให้ผลลัพธ์ตามจุดประสงค์ของการทดสอบจะต้องทำงาน}
\end{itemize}
\textbf{ขั้นตอนการทดสอบการลัดวงจร}
\begin{itemize}
{\item ขั้นแรกสวิตซ์ตัวนำต่างๆที่ใช้สำหรับการชาร์จและดิสชาร์จต้องปิดวงจรเพื่อจำลองถึงการใช้งานแบตเตอรี่ขณะขับขี่ยานยนต์ไฟฟ้าและการชาร์จแบตเตอรี่ภายนอกยานยนต์ไฟฟ้าถ้าหากขั้นตอนนี้ไม่สำเร็จให้ทำขั้นตอนนี้อีกครั้งจนกว่าจะสำเร็จ}
{\item ขั้วบวกและขั้วลบของแบตเตอรี่จะต้องทำการเชื่อมต่อถึงกันและกันเพื่อให้เกิดการลัดวงจรโดยอุปกรณ์การเชื่อมต่อนี้จะต้องมีความต้านทานไม่เกิน 5 มิลลิโอห์ม}
{\item การลัดวงจรจะถูกดำเนินไปอย่างต่อเนื่องจนกว่าจะถูกขัดจังหวะจากการทำงานของแบตเตอรี่หรือมีการจำกัดกระแสลัดวงจร หรือต้องมีการวัดอุณหภูมิที่ตัวแบตเตอรี่เป็นเวลาอย่างน้อย 1 ชั่วโมงโดยตลอดระยะเวลาที่ทำการวัดอุณหภูมิต้องมีการเปลี่นแปลงไม่เกิน 4$^{\circ}C$}
{\item การทดสอบจะยุติลงหลังจากการสังเกตการแบตเตอรี่ที่อุณหภูมิตามเงื่อนไขข้างต้นตามสภาพแวดล้อมที่ใช้ในการทดสอบ}
\end{itemize}
%=======================================================================================================
\subsection{การทดสอบการป้องกันการชาร์จเกินของแบตเตอรี่}
สำหรับหัวข้อการทดสอบนี้จะเป็นการทดสอบการป้องกันการชาร์จไฟฟ้าเกินขีดจำกัดของแบตเตอรี่เพื่อเป็นการทดสอบประสิทธิภาพการป้องกันการชาร์จเกินขีดจำกัดของแบตเตอรี่
\newline
\newline
\textbf{เงื่อนไขทั่วไปในขั้นตอนการทดสอบ}
\begin{itemize}
{\item ระหว่างการทดสอบแบตเตอรี่ต้องทำงานอยู่ในอุณหภูมิ 20$\pm$10$^{\circ}C$ หรือสูงกว่า}
{\item เมื่อเริ่มทำการทดสอบอุปกรณ์ป้องกันทุกอย่างที่ส่งผลต่อการทำงานของแบตเตอรี่ซึ่งให้ผลลัพธ์ตามจุดประสงค์ของการทดสอบจะต้องทำงาน}
\end{itemize}
\textbf{ขั้นตอนการทดสอบการชาร์จ}
\begin{itemize}
{\item ขั้นแรกสวิตซ์ตัวนำต่างๆที่ใช้สำหรับการชาร์จต้องปิดวงจร}
{\item อุปกรณ์ควบคุมจำกัดการชาร์จของอุปกรณ์วัดหรืออุปกรณ์ทดสอบแบตเตอรี่ต้องถูกปิดการใช้งาน}
{\item แบตเตอรี่ต้องถูกชาร์จด้วยอัตรากระแสอย่างน้อย 1/3 C แต่ต้องไม่เกินกระแสสูงสุดในช่วงการทำงานปกติตามที่ผู้ผลิตแบตเตอรี่ได้กำหนดไว้}
{\item การชาร์จจะถูกดำเนินไปอย่างต่อเนื่องจนกว่าการชาร์จจะถูกขัดจังหวะจากการทำงานของแบตเตอรี่หรือการชาร์จถึงขีดจำกัด เมื่อการขัดจังหวะโดยการทำงานของแบตเตอรี่นั้นไม่ทำงานหรือตัวแบตเตอรี่ไม่มีการทำงานในส่วนของการขัดจังหวะนี้การชาร์จจะถูกดำเนินต่อไปเรื่อยๆจนกว่าจะชาร์จถึง 2 เท่าของความจุพิกัด}
{\item การทดสอบจะยุติลงหลังจากการสังเกตการแบตเตอรี่ที่อุณหภูมิตามเงื่อนไขข้างต้นตามสภาพแวดล้อมที่ใช้ในการทดสอบ}
\end{itemize}
%=======================================================================================================
\subsection{การทดสอบการป้องกันการดิสชาร์จเกินของแบตเตอรี่}
ในการทดสอบการป้องกันการดิสชาร์จเกินโดยวัตถุประสงค์ของการทดสอบนี้เพื่อทดสอบความสามารถในการป้องกันการดิสชาร์จเกินของแบตเตอรี่โดยถ้าแบตเตอรี่มีอุปกรณ์ป้องกันการชาร์จเกินอยู่ภายในดังนั้นอุปกรณ์ป้องกันการชาร์จเกิน
นี้ต้องขัดจังหวะหรือจำกัดกระแสการดิสชาร์จเพื่อป้องกันความเสียหายต่างๆเนื่องจากค่า SOC ที่ต่ำเกินกว่าที่ผู้ผลิตแบตเตอรี่ได้กำหนดเอาไว้
\newline
\newline
\textbf{เงื่อนไขทั่วไปในขั้นตอนการทดสอบ}
\begin{itemize}
{\item ระหว่างการทดสอบแบตเตอรี่ต้องทำงานอยู่ในอุณหภูมิ 20$\pm$10$^{\circ}C$ หรือสูงกว่า}
{\item เมื่อเริ่มทำการทดสอบอุปกรณ์ป้องกันทุกอย่างที่ส่งผลต่อการทำงานของแบตเตอรี่ซึ่งให้ผลลัพธ์ตามจุดประสงค์ของการทดสอบจะต้องทำงาน}
\end{itemize}
\textbf{ขั้นตอนการทดสอบการดิสชาร์จ}
\begin{itemize}
{\item ขั้นแรกสวิตซ์ตัวนำต่างๆที่ใช้สำหรับการดิสชาร์จต้องปิดวงจร}
{\item อุปกรณ์ควบคุมจำกัดการชาร์จของอุปกรณ์วัดหรืออุปกรณ์ทดสอบแบตเตอรี่ต้องถูกปิดการใช้งาน}
{\item แบตเตอรี่ต้องถูกดิสชาร์จด้วยอัตรากระแสอย่างน้อย 1/3 C แต่ต้องไม่เกินกระแสสูงสุดในช่วงการทำงานปกติตามที่ผู้ผลิตแบตเตอรี่ได้กำหนดไว้}
{\item การดิสชาร์จจะถูกดำเนินไปอย่างต่อเนื่องจนกว่าการดิสชาร์จจะถูกขัดจังหวะจากการทำงานของแบตเตอรี่หรือการดิสชาร์จถึงขีดจำกัด เมื่อการขัดจังหวะโดยการทำงานของแบตเตอรี่นั้นไม่ทำงานหรือตัวแบตเตอรี่ไม่มีการทำงานในส่วนของการขัดจังหวะนี้การดิสชาร์จจะถูกดำเนินต่อไปเรื่อยๆจนกว่าแบตเตอรี่จะถูกดิสชาร์จจนถึง 25\% ของระดับแรงดันปกติ}
{\item หลังหยุดการดิสชาร์จแล้วแบตเตอรี่จะต้องนำไปชาร์จใหม่ด้วยอัตรากระแสปกติตามที่ผู้ผลิตได้กำหนดไว้ถ้าหากไม่ได้มีการกำหนดจะต้องทำการชาร์จด้วยอัตรากระแส 1/3 C}
{\item การทดสอบจะยุติลงหลังจากการสังเกตการแบตเตอรี่ที่อุณหภูมิตามเงื่อนไขข้างต้นตามสภาพแวดล้อมที่ใช้ในการทดสอบ}
\end{itemize}
%========================================================================================================
โดยทั้ง 3 หัวข้อของการทดสอบตามมาตรฐาน UN ECE Regulation 136 นั้นเงื่อนไขที่จะผ่านการทดสอบแบตเตอรี่มีดังนี้
\begin{enumerate}
{\item ในระหว่างการทดสอบแบตเตอรี่จะต้องไม่มีอิเล็กโทรไลต์รั่วไหลออกจากแบตเตอรี่ โดยการสังเกตการรั่วไหลของอิเล็กโทรไลต์ให้สังเกตโดยรอบของแบตเตอรี่เพียงเท่านั้นโดยไม่ต้องแยกชิ้นส่วนใดๆของแบตเตอรี่ออก}
{\item ในระหว่างการทดสอบแบตเตอรี่จะต้องไม่เกิดการแตกหักหรือฉีกขาด}
{\item ในระหว่างการทดสอบแบตเตอรี่จะต้องไม่เกิดเพลิงไหม้}
{\item ในระหว่างการทดสอบแบตเตอรี่จะต้องไม่เกิดการระเบิด}
\end{enumerate}
%========================================================================================================
\section{อุปกรณ์สำหรับทดสอบแบตเตอรี่}
ในการทดสอบแบตเตอรี่สำหรับโครงงานนี้อุปกรณ์หลักที่จะใช้ในการทดสอบในหัวข้อต่างๆคือเครื่องทดสอบแบตเตอรี่ Chroma Model 17020 โดยใช้งานเครื่องทดสอบแบตเตอรี่ในการทดสอบแบตเตอรี่ในแต่ละหัวข้อมีดังนี้
\subsection{การใช้เครื่องทดสอบแบตเตอรี่ Chorma Model 17020 \\ การทดสอบการป้องกันการลัดวงจรภายนอกของแบตเตอรี่}
















