% This file requires some necessary information used in a writing.
% Some optional commands, mark with "OPT", can be omitted by making them as comments 
% (type "%" before them).

\pagestyle{topright}

% PART 1: About author and thesis info.
% ==============================================

% Thesis title for cover and other related pages
\thesistitleThai{ชื่อเรื่องวิทยานิพนธ์}
\thesistitle{Thesis or Dissertation Title}

% Thesis title for the abstract page
\thesistitleforabstractThai{ชื่อเรื่องวิทยานิพนธ์}
\thesistitleforabstract{Thesis or Dissertation Title}

% Name and surename of a student
%\authorThai1{นายณัฐนนท์ กาญจนประภาส}
%\authorThai2{นายณัฐนนท์ กาญจนประภาส}
\author{Firstname Surename}

% Name title of a student
\nametitle{คำนำหน้าชื่อ}

%% Date of birth of a student
%\dateofbirth{ชื่อเต็มของเดือน พ.ศ.}

%% Work position of a student while studying OPT
%\workposition{ชื่อตำแหน่งงานปัจจุบัน สังกัด} 

%% Educational attainment befor this study
%\eduattainment{ปีการศึกษา 25xx: ชื่อปริญญา\\
%	ชื่อมหาวิทยาลัย
%}

%% Scholarship OPT
%\scholarship{ปีพ.ศ.เรียงจากใหม่ไปหาเก่า: ชื่อทุนการศึกษาที่ได้รับ}

%% Work experiences OPT
%\workexperiences{ปีพ.ศ.เรียงจากใหม่ไปหาเก่า: ชื่อตำแหน่งงาน\\
%	สถานที่ทำงาน
%}

% Degree title of this study
\degreeThai{ชื่อเต็มปริญญาที่ได้รับ}
\degree{Degree Title}

% Major field of this study OPT
\majorThai{ชื่อสาขาวิชา}
\major{Major Field}

% Department OPT
\departmentThai{ภาควิชา}
\department{department}

% Faculty
\facultyThai{คณะ}
\faculty{Faculty}

% Academic year that this thesis is submitted
\academicyear{2564}


% Fill a type of writing, e.g., Independent Study, Thesis or Dissertation.
\typeofwritingThai{วิทยานิพนธ์}
\typeofwriting{Thesis}

% Approval date
\approvaldate{วันที่ ชื่อเต็มของเดือน พ.ศ. 2564}



% PART 2: Examination Committee with their academic degrees
% ==============================================

% Chairman
\chairman{ชื่อตำแหน่งทางวิชาการ ชื่อ ชื่อสกุลอาจารย์}
%\chairmandegree{Ph.D./M.D.}

% Advisor
\advisorThai{ชื่อตำแหน่งทางวิชาการ ชื่อ ชื่อสกุลอาจารย์}
\advisor{Academic Title Firstname Surname}
%\advisordegree{Ph.D./M.D.}

% Co-advisor OPT
\coadvisorThai{ชื่อตำแหน่งทางวิชาการ ชื่อ ชื่อสกุลอาจารย์}
\coadvisor{Academic Title Firstname Surname}
%\coadvisordegree{Ph.D./M.D.}

% Other member
\memberone{ชื่อตำแหน่งทางวิชาการ ชื่อ ชื่อสกุลอาจารย์}
%\memberonedegree{Ph.D./M.D.}

\membertwo{ชื่อตำแหน่งทางวิชาการ ชื่อ ชื่อสกุลอาจารย์} %OPT
%\membertwodegree{Ph.D./M.D.} %OPT

% Dean of faculty
\dean{ชื่อตำแหน่งทางวิชาการ ชื่อ ชื่อสกุลอาจารย์}
%\deandegree{Ph.D./M.D.}




% PART 3: Abstract of this writing and keywords
% ==============================================

\begin{abstractThai}
	โครงงานนี้ได้นำเสนอวิธีการใช้เครื่องทดสอบแบตเตอรี่ Chroma 17020 ในการทดสอบแบตเตอรี่ตามมาตรฐาน UN ECE Regulation 136 ในหัวข้อการทดสอบการป้องกันการดิสชาร์จเกินและ
	การทดสอบการป้องกันการชาร์จเกินและใช้ทดสอบอื่นๆเช่นการทดสอบการวัดความต้านทานภายในของแบตเตอรี่ ซึ่งแบตเตอรี่สำหรับทดสอบในหัวข้อต่างๆมีด้วยกัน 2 ชนิดคือแบตเตอรี่ลิเธียม NMC และแบตเตอรี่ lifepo4
	สำหรับแบตเตอรี่ลิเธียม NMC ที่ใช้ในการทดสอบได้แก่ แบตเตอรี่สำหรับจักรยานยนต์ไฟฟ้า 72V30Ah และแบตเตอรี่สำหรับรถสามล้อไฟฟ้า 72V60Ah สำหรับแบตเตอรี่ lifepo4 ที่ใช้ทดสอบได้แก่
	แบตเตอรี่ 72V72Ah โดยการทดสอบนี้เป็นแนวทางการทดสอบตามมาตรฐานเบื้องต้นเพื่อนำไปใช้สำหรับยานยนต์ไฟฟ้า
\end{abstractThai}

\keywordsThai{การทดสอบแบตเตอรี่สำหรับยานยนต์ไฟฟ้า, การทดสอบการป้องกันการชาร์จเกิน,\\การทดสอบการป้องกันการดิสชาร์จเกิน}

\begin{abstract}
	This project presents a method of testing a battery packs with the Chroma 	
	Model 17020 Regenerative Battery Pack Test System.The tests are based on UN ECE \mbox{Regulation} 136 in section the Overcharge potection test and the Overdischarge potection test.The other testing are relaxation time of a battery pack and a c-rate test that how it is affect a battery pack.The testing is conducted on a serveral
	battery packs which is the Li-ion NMC 72V30Ah battery pack for e-motorcycle, the Li-ion NMC 72V60Ah battery      pack for e-tricycle lastly The LiFePO4 72V72Ah	battery pack.
	
\end{abstract}  

\keywords{Lithium-ion battery pack, Battery pack testing,LiFePO4 battery pack}




% PART 4: Acknowledgements
% ==============================================
\begin{acknowledgements}
	โครงงานวิศวกรรมการทดสอบแบตเตอรี่สำหรับยานยนต์ไฟฟ้า สามารถดำเนินงานและจัดโครงงานสำเร็จได้ด้วยดี เพราะได้รับความอนุเคราะห์จากอาจารย์ ดร.วสันต์ ตันเจริญ ซึ่งเป็นอาจารย์ที่ปรึกษาโครงงาน โดยท่านได้ให้แนวคิดและข้อมูลบางส่วนของโครงงานนี้อีกทั้งยังคอยให้คำปรึกษาแนะนำในการดำเนินงานตลอดจนให้คำตักเตือนเกี่ยวกับข้อผิดพลาดโดยคณะผู้จัดทำสามารถการแก้ไขปัญหาต่างๆ จนสำเร็จตามเป้าหมาย
 
ขอกราบขอบพระคุณท่านคณาจารย์ประจำภาควิชาวิศวกรรมไฟฟ้าและอิเล็กทรอนิกส์ คณะวิศวกรรมศาสตร์ศรีราชา มหาวิทยาลัยเกษตรศาสตร์ วิทยาเขตศรีราชาทุกท่านที่ประสิทธิ์ประสาทวิชาความรู้เพื่อให้นำความรู้ที่ได้เรียนมาใช้ในการทำโครงงานนี้ ขอขอบคุณ ช่างเทคนิคนาย วสันต์ สินธุยศ ช่างเทคนิคชำนาญงานนาย คมสัน สุนันท์รุ่งอังคณา ที่คอยให้คำแนะนำและข้อมูลต่างๆที่เป็นประโยชน์ต่อโครงงานนี้ทำให้โครงงานสามารถสำเร็จลุล่วงได้ด้วยดี

สุดท้ายนี้ขอกราบขอบพระคุณบิดา มารดา ผู้ที่ให้กำเนิด และอบรมสั่งสอนจนเติบใหญ่ รวมถึงเพื่อน พี่ น้อง ที่คอยช่วยเหลือในการหาข้อมูลสนับสนุนต่างๆ ประโยชน์อันใดที่เกิดจากการทำโครงงานนี้ย่อมเป็นผลมาจากความกรุณาของท่านดังกล่าวข้างต้น ผู้จัดทำโครงงานรู้สึกซาบซึ้งอย่างยิ่งและขอขอบพระคุณอย่างสูงไว้ ณ โอกาสนี้
\begin{flushright}
\begin{tabular}{@{}c@{}}
นายณัฐนนท์ กาญจนประภาส\\
นายณัฐนันท์ อุบลวัจ
\end{tabular}
\end{flushright}

\end{acknowledgements}



% PART 5: Your Publications
% ==============================================
\begin{publications}
	ชื่อผลงานทางวิชาการ (ลงรายการอ้างอิง)
\end{publications}



% PART 6: List of Abbreviations
% ==============================================
% Symbols
\nomenclature[1ph]{$ \varphi $}{A Greek alphabet}
\nomenclature[1ps]{$ \psi $}{An other Greek alphabet}

% Alphabets
\nomenclature[2B]{$ \mathbf{B}(X,Y) $}{The set of all bounded linear operator from $ X $ to $ Y $ the set of all bounded linear operator from $ X $ to $ Y $}
\nomenclature[2R]{$ \mathbb{R} $}{The set of real numbers}
\nomenclature[2R]{R}{The 18th of English alphabets}




